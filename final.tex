\documentclass{beamer} 

\mode<presentation>{\usetheme{Warsaw}} 
\usefonttheme{serif}

\title{Rachmaninoff --- The Bells, op. 35} 
\subtitle{4. Lento lugubre} 
\author{Kalvin Lee} 
\date{} 

\begin{document} 

\begin{frame} 
  \titlepage
\end{frame} 

\begin{frame} 
  If you haven't yet read the 30-second Rachmaninoff primer provided with this presentation, please take a moment to do so now. It'll help introduce you to this fascinating composer and this exemplary composition. 
\end{frame} 

\begin{frame}
  The movement begins with key-setting: heavily falling fifths in C sharp minor. 
  \pause \\ 
  A reeded solo leads, drawing attention to the seventh (C\# \(\to\) \emph{B}) 
  \pause \\
  Chromaticism colors what began in natural minor mode --- the uncanny ``slithering'' plays to unnerve. 
  \pause \\ 
  The slow, plodding funeral march is interrupted sharply in A minor, \pause before carrying on, the seventh now mingling freely with the second. (The second creates an open-ended sound, deliberately avoiding resolution like what the tonic would produce). 
  \pause \\ 
  Another interruption --- this time in F minor. 
  \pause \\ 
  The solo carries on undeterred, \pause even by another interruption in D minor. 
  \pause \\ 
  This whole time, the reed solo has been unwinding the first theme --- the jump to the seventh and the serpentine chromaticism are its signatures. 
\end{frame} 

\begin{frame}
  At last the soloist enters, carried in by a half-diminished chord: 
  \pause \\ 
  \begin{quote} 
    Hear{\pause} the{\pause} toll{\pause}ing{\pause} of{\pause} the{\pause} bells,{\pause} mourn{\pause}ful{\pause} bells.{\pause} 
  \end{quote} 
  Chattering low strings stir uneasily. The basses strum lightly like pallbearers' footsteps. 
  \pause \\ 
  The choir contributes a solemn, rhythmically smooth backing to the dirge. 
  \pause \\ 
  The reeded first theme continues as a countermelody behind the soloist. 
  \pause \\ 
  The sharpest interruption yet, in F sharp minor. 
  \pause \\ 
  Again, the ``slithering'' sound --- an idea, just a fragment of the (first) theme. 
\end{frame} 

\begin{frame}
  The ``slithering'' first theme's chromatic movements are still audible. 
  \pause \\ 
  The interruption, in hushed tones, in D minor --- \pause then in F minor. 
  \pause \\ 
  Dreadfully rising figures between the soloist and the whole choir, like the cries of a mourning procession: 
  \pause \\
  \begin{quote} % Glad endeavour quenched forever in the silence and the gloom. 
    G{\pause}l{\pause}a{\pause}d{\pause} e{\pause}n{\pause}d{\pause}e{\pause}a{\pause}v{\pause}o{\pause}u{\pause}r{\pause} q{\pause}u{\pause}e{\pause}n{\pause}c{\pause}h{\pause}e{\pause}d{\pause} f{\pause}o{\pause}r{\pause}e{\pause}v{\pause}e{\pause}r{\pause} i{\pause}n{\pause} t{\pause}h{\pause}e{\pause} s{\pause}i{\pause}l{\pause}e{\pause}n{\pause}c{\pause}e{\pause} a{\pause}n{\pause}d{\pause} t{\pause}h{\pause}e{\pause} g{\pause}l{\pause}o{\pause}o{\pause}m{\pause}. 
  \end{quote} 
\end{frame} 

\begin{frame}
  The sevenths still dog the soloist and the leading reed. The latter retains the ``slithering'' chromaticism. 
  \pause \\ 
  The soloist steers the choir and orchestra into a rising, development-like section that grows in intensity --- 
  \pause \\ 
  louder 
  \pause \\ 
  and louder 
  \pause \\ 
  \textellipsis until it dies out. 
\end{frame} 

\begin{frame}
  The head of the \textit{Dies Irae} can be heard in bits and pieces; \pause four notes at a time, \pause then two, almost unrecognizable. 
  \pause \\ 
  The celestia hints dreamily at the same. 
  \pause \\ 
  The strings stir anxiously again in anticipation of new material. 
  \pause \\ 
  The first theme again (notice the seventh and the slithering) over the stirring. \pause The \textit{Dies Irae} comes on its heels. 
  \pause \\ 
  The interruption in F minor again. Here goes something. 
  \pause \\ 
  Unceasingly, the \textit{Dies Irae} still darts about in the lower registers. 
  \pause \\ 
  A moment of silence gives way to a solo line followed by an ``interruption.'' 
  \pause \\ 
  The ``slithering'' sound of the primary theme can be heard in the strings, tinted by an air of sharpened anticipation.
  \pause \\ 
  In this moment, the first theme and the \textit{Dies Irae} manage to coexist in the same musical frame. 
\end{frame} 

\begin{frame}
  The first theme is heard again in the rising seventh and the ``slithering'' motion in the brass. Note also the lower, more menacing tip-toe sounding of the \textit{Dies Irae}. 
  \pause \\ 
  The soloist is punctuated with brass. 
  \pause \\ 
  The descending contour of the \textit{Dies Irae} is reflected in the vocal line. 
  \pause \\ 
  The choir, too, sounds with echoes of the \textit{Dies Irae}. 
  \pause \\
  No longer an echo, it blossoms into a terrible and full statement. 
  \pause \\ 
  The terrible climax, the horrible cries, soaring over the biggest statement yet of the chant. 
\end{frame} 

\begin{frame}
  The climactic passage fades on open and non-resolving harmonies. 
  \pause \\
  The choir hums in tenderly falling thirds, harking back to the ``humming'' middle section of the first movement (``The Silver Sleigh Bells''). It is some coincidence (?) that the humming in the first movement is in the same key as this movement. 
  \pause \\
  The soloist returns on the first theme with greater rhythmic freedom than began the movement, sung over thicker and more tender string writing. 
\end{frame} 

\begin{frame}
  The orchestral postlude, in the enharmonic parallel major of D flat major. 
  \pause \\
  The rising seventh (descended from the octave), heard one last time in the major key. 
  \pause \\
  The movement (and the symphony) concludes with an unconventional plagal cadence (subdominant \(\to\) tonic)\footnote{The plagal cadence cowers before the authentic cadence as a resolution; yet its frequent use as an ``Amen'' in many settings have lent it a contrived religious serenity.}. 
\end{frame} 


\end{document} 
