\documentclass{beamer} 

\mode<presentation>{\usetheme{Warsaw}} 
\usefonttheme{serif}

\title{Rachmaninoff --- The Bells, op. 35} 
\subtitle{4. Lento lugubre} 
\author{Kalvin Lee} 
\date{} 

\begin{document} 

\begin{frame} 
  \titlepage
\end{frame} 

\begin{frame}
  The movement begins with key-setting: heavily falling fifths in C sharp minor. 
  \pause \\ 
  A reeded solo leads, highlighting the motion to the seventh (C\# \(\to\) \emph{B}) 
  \pause \\ 
  Although this is the final movement of the symphony, per the Poe text it drags slowly and sadly to its conclusion. Tchaikovsky already covered this base in his sixth symphony, but none of Rachmaninoff's other symphonic works do this. 
  \pause \\
  Chromaticism colors what began in natural minor mode --- the uncanny ``slithering'' plays to unnerve. 
\end{frame} 

\begin{frame} 
  The slow, plodding funeral march is interrupted sharply in A minor, \pause before carrying on, the seventh now mingling freely with the second. 
  \pause \\
  The second creates an open-ended sound, deliberately avoiding resolution like what the tonic would produce. 
  \pause \\ 
  Another interruption --- this time in F minor. 
  \pause \\ 
  The solo carries on undeterred, \pause even by another interruption in D minor. 
  \pause \\ 
  This whole time, the reed solo has been unwinding the first theme --- the jump to the seventh and the serpentine chromaticism are its signatures. Additionally, the idea of the attention-drawing ``interruption'' will recur as the movement goes on. 
\end{frame} 

\begin{frame}
  At last the soloist enters, carried in by a half-diminished chord. The entrance is melodically the flattest vocal line in the whole symphony: a repeated C sharp: 
  \pause \\ 
  \begin{quote} 
    Hear{\pause} the{\pause} toll{\pause}ing{\pause} of{\pause} the{\pause} bells,{\pause} mourn{\pause}ful{\pause} bells.{\pause} 
  \end{quote} 
  Chattering low strings stir uneasily. The basses strum lightly like pallbearers' footsteps. 
  \pause \\ 
  The choir contributes a solemn, rhythmically smooth backing to the dirge. 
  \pause \\ 
  The reeded first theme continues as a countermelody behind the soloist. 
  \pause \\ 
  The sharpest interruption yet, in F sharp minor. 
  \pause \\ 
  Again, the ``slithering'' sound --- an idea, just a fragment of the (first) theme. 
\end{frame} 

\begin{frame}
  The ``slithering'' first theme's chromatic movements are still audible. 
  \pause \\ 
  The interruption, in hushed tones, in D minor --- \pause then in F minor. 
  \pause \\ 
  Dreadfully rising figures pass between the soloist and the whole choir, like the cries of a mourning procession: 
  \pause 
  \begin{quote} % Glad endeavour quenched forever in the silence and the gloom. 
    G{\pause}l{\pause}a{\pause}d{\pause} e{\pause}n{\pause}d{\pause}e{\pause}a{\pause}v{\pause}o{\pause}u{\pause}r{\pause} q{\pause}u{\pause}e{\pause}n{\pause}c{\pause}h{\pause}e{\pause}d{\pause} f{\pause}o{\pause}r{\pause}e{\pause}v{\pause}e{\pause}r{\pause} i{\pause}n{\pause} t{\pause}h{\pause}e{\pause} s{\pause}i{\pause}l{\pause}e{\pause}n{\pause}c{\pause}e{\pause} a{\pause}n{\pause}d{\pause} t{\pause}h{\pause}e{\pause} g{\pause}l{\pause}o{\pause}o{\pause}m{\pause}. 
  \end{quote} 
\end{frame} 

\begin{frame}
  Note that the movement is not cleanly delimited between sections; the transitions are subtle and often only signalled through changes of rhythm. Thus the whole movement flows seamlessly from start to finish (contrast against the ``halting'' deployed in the first movement of Schubert's eighth symphony to transition from the stormy minor theme to the pretty major theme). 
\end{frame} 

\begin{frame}
  The sevenths still dog the soloist and the leading reed. The latter retains the ``slithering'' chromaticism. Strumming bass strings proceed under the reprise of the first theme in the vocal line. 
  \pause \\ 
  The soloist steers the choir and orchestra into a rising, development-like section that grows in intensity --- louder and louder and louder still, until it reaches its apex and dies out. 
  \pause \\ 
  Thus far, the movement has not presented any especially ``singable'' theme that a concertgoer would go home humming. \emph{Ideas} dominate the movement: the rising seventh, the ``slithering'' motion, the sforzando interruption \textellipsis we will encounter the next big idea shortly. 
\end{frame} 

\begin{frame}
  In the whirling eddies of wind left behind by the preceding section, the head of the \textit{Dies Irae} can be heard in bits and pieces; \pause four notes at a time, \pause then two, almost unrecognizable. 
  \pause \\
  This will function as the second theme of the movement, albeit smeared craftily across a huge swath of the music. 
  \pause \\ 
  Chromaticism is no longer the sole domain of the reed or the vocalist --- it sneaks about within the texture below. 
  \pause \\ 
  The celestia hints dreamily at the \textit{Dies Irae}. 
  \pause \\ 
  The two themes complement each other --- the first theme is a solemn processional, whereas the \textit{Dies Irae} is pathos-laden and emotionally charged. 
  \pause \\ 
  The strings stir anxiously again in anticipation of new material. 
\end{frame} 

\begin{frame} 
  The first theme again (notice the seventh and the slithering) over the stirring. \pause The \textit{Dies Irae} comes on its heels. 
  \pause \\ 
  The interruption in F minor again. Here goes something. 
  \pause \\ 
  Unceasingly, the \textit{Dies Irae} still darts about in the lower registers. 
  \pause \\ 
  A moment of silence gives way to a solo moment followed by an ``interruption.'' 
  \pause \\ 
  The first theme is heard again in the rising seventh and the ``slithering'' motion as well. Note also the lower, more menacing tip-toe sounding of the \textit{Dies Irae}. 
  \pause \\ 
  In this moment, the first theme and the \textit{Dies Irae} manage to coexist in the same musical frame. This reflects a changing of the guard --- that the stately procession is soon to give way to raw emotion. 
\end{frame} 

\begin{frame}
  The soloist ushers in a new contrasting section; the music is driven briskly along. 
  \pause \\ 
  The descending contour of the \textit{Dies Irae} is reflected in the vocal line. 
  \pause \\ 
  The choir, too, sounds with echoes of the \textit{Dies Irae}; it passes from register to register. 
  \pause \\
  No longer an echo, it blossoms into a terrible and full statement. 
  \pause \\ 
  The terrible climax, the horrible cries, soaring over the biggest statement yet of the chant. 
  \pause \\ 
  This is a not-so-straightforward application of the \textit{Dies Irae} that flits around in little segments and extracts, transformed sometimes almost beyond recognition. 
\end{frame} 

\begin{frame}
  The climactic passage fades on open and non-resolving harmonies. 
  \pause \\
  The choir hums in tenderly falling thirds, harking back to the ``humming'' middle section of the first movement (``The Silver Sleigh Bells''). It is some coincidence (?) that the humming in the first movement is in the same key as this movement. 
  \pause \\
  This is the penultimate section of the symphony: we have traversed the dark and sad opening, the frenzied and horrifying middle section, we now enter the cool-off, ready to segue away to the very end. 
  \pause \\
  The soloist returns on the first theme with greater rhythmic freedom than began the movement, sung over thicker and more tender string writing. The sound of reeds is altogether absent as a countermelody; the ``slithering,'' too, is missing. 
  \pause \\ 
  The saddest part of the funeral is over: after shock and mourning come reflections and healing. 
\end{frame} 

\begin{frame}
  The calming orchestral postlude, in the enharmonic parallel major of D flat major. 
  \pause \\
  The rising seventh (descended from the octave) is heard one last time in the major key. 
  \pause \\
  Again, nothing is easily ``singable,'' not before and not now. The music is clear in its delivery, but its slightly unconventional harmonies (among other things) make it a little less memorable. 
  \pause \\ 
  In the wake of the rest of this tragic movement, the modulation gives the listener a particularly sweet happy cure. It is more honest than the pompous and emphatically jovial conclusions to its siblings in the second and third symphonies (predecessor and successor, respectively). 
  \pause \\
  The movement (and the symphony) concludes with an unconventional plagal cadence (subdominant \(\to\) tonic)\footnote{The plagal cadence cowers before the authentic cadence as a resolution; yet its frequent use as an ``Amen'' in many settings have lent it a contrived religious serenity.}. 
\end{frame} 

\begin{frame}
  \begin{center} 
    {\Large\textsc{The End} } 
  \end{center} 
\end{frame} 


\end{document} 
