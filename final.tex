\documentclass{beamer} 

\mode<presentation>{\usetheme{Warsaw}} 
\usefonttheme{serif}
\usepackage{graphicx} 

\title{Rachmaninoff --- The Bells, op. 35} 
\subtitle{4. Lento lugubre} 
\author{Kalvin Lee} 
\date{} 

\begin{document} 

\begin{frame} 
  \titlepage
\end{frame} 

\begin{frame} 
  \begin{center} 
    \includegraphics[width=9.1cm]{authors} \\ 
    {\small Poe, Balmont, and Rachmaninoff. } 
  \end{center} 
\end{frame} 

\begin{frame}
  The movement begins with key-setting: heavily played fifths in C sharp minor. 
  \pause \\ 
  A reed solo leads, highlighting the motion to the seventh (C\# \(\to\) \emph{B}), descending slowly from there. This is more or less the first theme of the movement, but be careful about being attached to it, because the music is very loosely thematically unified. 
\end{frame}

\begin{frame} 
  \begin{center} 
    \includegraphics[height=3.9cm]{tchaikovsky}
  \end{center} 
  Although this is the final movement of the symphony, per the Poe text it drags slowly and sadly to its conclusion. Tchaikovsky already covered this base in his sixth symphony, but none of Rachmaninoff's other symphonic works do this. 
  \pause \\
  Chromaticism colors what began in natural minor mode --- the uncanny ``slithering'' plays to unnerve (rather than to \emph{sing}, as Tchaikovsky would have it). 
\end{frame} 

\begin{frame} 
  The slow, plodding funeral march is interrupted sharply in A minor, \pause before carrying on, \pause the same rising seventh now slowly making its descent towards the second (D\#). 
  \pause \\
  The second creates an open-ended sound, deliberately avoiding resolution like what the tonic would produce. 
  \pause \\ 
  This paradigm of leaving sections open-ended will continue through the movement, lending the symphony a rhapsodic flow. This is anomalous, even for Rachmaninoff, whose other large-scale works were carefully structured. 
\end{frame}

\begin{frame} 
  Another interruption --- this time in F minor. 
  \pause \\ 
  The solo carries on undeterred, modulating without apology back to C sharp minor. 
  \pause \\ 
  Another interruption --- this one in D minor. 
  \pause \\ 
  By now, the reed solo has familiarized us with the main ``theme''  --- the jump to the seventh and the serpentine chromaticism are its signatures. Additionally, the idea of the attention-drawing ``interruption'' will recur as the movement goes on. 
\end{frame} 

\begin{frame}
  \begin{center} 
    \includegraphics[height=3.9cm]{dinging}
  \end{center} 
  At last the soloist enters, carried in by a half-diminished chord. The entrance is melodically the flattest vocal line in the whole symphony: a repeated C sharp over a downward woodwind spiral: 
  \pause 
  \begin{quote} 
    Hear{\pause} the{\pause} toll{\pause}ing{\pause} of{\pause} the{\pause} bells,
  \end{quote} 
\end{frame}

\begin{frame} 
  The flatness of the entrance stands at odds with the standard set by Tchaikovsky, Rachmaninoff's old teacher, and with Rachmaninoff himself, who usually broke out the melodic warhorses for the finales of his large-scale compositions. 
  \pause \\ 
  Chattering low strings stir uneasily. The bass strings strum lightly like pallbearers' footsteps. 
  \pause 
  \begin{quote} 
    mourn{\pause}ful{\pause} bells. 
  \end{quote} 
\end{frame} 

\begin{frame} 
  The choir contributes a solemn, rhythmically smooth backing to the dirge.
  \pause \\ 
  The reeded first theme continues as a countermelody behind the soloist. Note the rising seventh and the slimy chromatics again. This shared center of attention evokes a little bit of the double concerto tradition. (The whole symphony focuses the three soloists --- tenor, soprano, and baritone, respectively --- in turn, so no single singer dominates.) 
  \pause \\ 
  The sharpest interruption yet, in F sharp minor. Again, the ``slithering'' sound --- an idea, just a fragment of the (first) theme. 
\end{frame} 

\begin{frame}
  What sounded a little like a modulation simply circles back to the same half-diminished chord at the vocalist's entrance. 
  \pause \\ 
  The ``slithering'' first theme's chromatic movements are still audible. 
  \pause \\ 
  The interruption, in hushed tones, in D minor --- \pause then in F minor. 
\end{frame} 

\begin{frame} 
  \begin{center} 
    \includegraphics[height=3.9cm]{calcified} 
  \end{center} 
  Dreadfully rising figures pass between the soloist and the whole choir, like the cries of a mourning procession: 
  \pause 
  \begin{quote} % Glad endeavour quenched forever in the silence and the gloom. 
    Glad{\pause} en{\pause}dea{\pause}vour{\pause} quenched{\pause} for{\pause}e{\pause}ver{\pause} in{\pause} the{\pause} si{\pause}lence{\pause} and{\pause} the{\pause} gloom. 
  \end{quote} 
\end{frame} 

\begin{frame}
  Note that the movement is not cleanly delimited between sections; the transitions are subtle and often only signaled through changes of rhythm. Thus the whole movement flows seamlessly from start to finish (contrast against the ``halting'' deployed in the first movement of Schubert's eighth symphony to transition from the stormy minor theme to the pretty major theme). 
\end{frame} 

\begin{frame}
  The sevenths still dog the soloist and the leading reed. 
  \pause \\ 
  The latter retains the ``slithering'' chromaticism. 
  \pause \\ 
  The soloist steers the choir and orchestra into a chromatically rising, development-like section that grows in intensity --- louder and louder and louder still, until it reaches its apex and dies out. 
\end{frame} 

\begin{frame} 
  \emph{Ideas}, rather than themes, continue to dominate the movement: the rising seventh, the ``slithering'' motion, the sforzando interruption \textellipsis we will encounter the next big idea shortly. 
  \pause \\ 
  If the atmosphere of the music has gotten to you, please remember that this is middle-period Rachmaninoff --- a sizable body of work follows \textit{The Bells}, including the third (numbered) symphony, the fourth piano concerto, the \textit{Paganini Rhapsody}, and the \textit{Symphonic Dances}. 
  \pause \\ 
  There was not some terrific internal struggle going on within the composer: he simply committed what he thought was appropriate, perhaps a little like Mozart going with the minor mode in vogue at the time of the 25th symphony. 
\end{frame} 

\begin{frame}
  In the whirling eddies of wind left behind by the preceding section, the head of the \textit{Dies Irae} can be heard in bits and pieces; \pause four notes at a time, \pause then two, almost unrecognizable. 
  \pause \\
  This will function as the second theme of the movement, albeit smeared craftily across a huge swath of the music. As mentioned in the included primer, Rachmaninoff was very fond of deploying the chant in his compositions; it appeared in his very first symphony and also in his last large-scale composition, the \textit{Symphonic Dances}. 
\end{frame} 

\begin{frame} 
  The vocalist performs a microscopic give-and-take with the sadly chirruping strings. 
  \pause \\ 
  The celesta hints dreamily at the \textit{Dies Irae}. 
  \pause \\ 
  The two ``themes'' complement each other --- the first theme is a solemn processional, whereas the \textit{Dies Irae} is pathos-laden and emotionally charged. 
  \pause \\ 
  The strings stir anxiously again, foretelling new material. 
\end{frame} 

\begin{frame} 
  \begin{center} 
    \includegraphics[height=3.9cm]{isle} 
  \end{center} 
  The first theme again (notice the seventh and the slithering) over the stirring. 
  \pause \\
  The \textit{Dies Irae} comes on its heels. 
\end{frame} 

\begin{frame} 
  This segment represents a not-so-straightforward application of the \textit{Dies Irae} that flits around in little segments and extracts, transformed sometimes almost beyond recognition. 
  \pause \\ 
  Stirring strings imbue the formerly tragic theme with anticipation. 
  \pause \\ 
  The interruption in F minor again. Here goes something. 
\end{frame}
  
\begin{frame} 
  Unceasingly, the \textit{Dies Irae} still darts about in the lower registers. 
  \pause \\ 
  A moment of silence gives way to a solo moment followed by two ``interruptions,'' one softer, one louder; the louder one hearkens back to the recurring chromatic shimmy. 
  \pause \\ 
  The first theme is heard again in the rising seventh and the ``slithering'' motion as well. Note also the lower, more menacing tip-toe sounding of the \textit{Dies Irae}. 
  \pause \\ 
  In this moment, the first theme and the \textit{Dies Irae} manage to coexist in the same musical frame. This reflects a changing of the guard --- that the stately procession is soon to give way to raw emotion. 
\end{frame} 

\begin{frame}
  The soloist ushers in a new contrasting section; interruptions heighten the sense of give-and-take. The music is driven briskly along. 
  \pause \\ 
  The descending contour of the \textit{Dies Irae} is reflected in the vocal line. 
  \pause \\
  The orchestra oscillates sharply on a minor third. 
  \pause \\ 
  The choir, too, sounds with echoes of the \textit{Dies Irae}; it passes from register to register. 
  \pause \\
  Fluttering winds create a pressing sense of urgency as the music gains momentum, starting to sound unstoppable. 
  \pause \\ 
  The falling thirds of the \textit{Dies Irae} weigh against a descending diminished fifth interval. These go between soloist and choir. 
\end{frame} 

\begin{frame}
  \begin{center} 
    \includegraphics[height=3.9cm]{ringing} 
  \end{center} 
  No longer an echo, it blossoms into a terrible and full statement: horrible cries soar over the biggest statement yet of the chant. 
  \pause \\ 
  Rachmaninoff pushes the orchestral envelope here for a positively demonic sound. 
\end{frame} 

\begin{frame}
  \begin{center} 
    \includegraphics[height=3.9cm]{maskedbells} 
  \end{center} 
  The climactic passage fades on open and non-resolving harmonies, the \textit{Dies Irae} still peeking out from between the descending arpeggios of seventh chords. 
  \pause \\ 
  This is the penultimate section of the symphony: we have traversed the dark and sad opening, the frenzied and horrifying middle section, we now enter the cool-off, ready to segue to the very end. 
  \pause \\ 
  Rachmaninoff continues writing modestly, focusing on letting the music cool off naturally. 
\end{frame} 

\begin{frame} 
  \begin{center} 
    \includegraphics[height=3.9cm]{funeral} 
  \end{center} 
  Arpeggiated seventh chords scatter the ashes of the preceding section away. 
  \pause \\
  The choir hums in tenderly falling thirds, harking back to the ``humming'' middle section of the first movement (``The Silver Sleigh Bells''). It is some coincidence (?) that the humming in the first movement is in the same key as this movement. 
\end{frame} 

\begin{frame} 
  The soloist returns on the first theme with greater rhythmic freedom than began the movement, sung over thicker and more tender string writing. 
  \pause \\ 
  The sound of reeds is altogether absent as a countermelody. 
  \pause \\ 
  The chromatic movement (formerly ``slithering'') is no longer foreboding, but takes on a sweet sound like the yearning singing from the second movement (``The Mellow Wedding Bells''). But unlike before, the descending chromatic movement is sad, not rapturous. 
  \pause \\ 
  The saddest part of the funeral is over: after shock and mourning come reflections and healing. 
\end{frame} 

\begin{frame}
  The calming orchestral postlude, in the enharmonic parallel major of D flat major. 
  \pause \\
  Again, nothing is easily ``singable,'' not before and not now. The music is clear in its delivery, but its slightly unconventional harmonies (among other things) make it a little less memorable. 
  \pause \\
  In the wake of the rest of this tragic movement, the modulation gives the listener a particularly sweet happy cure. It is more honest than the pompous and emphatically jovial conclusions to its siblings in the second and third symphonies (predecessor and successor, respectively). 
  \pause \\
  The rising seventh (descended from the octave) is heard one last time in the major key. 
\end{frame} 

\begin{frame} 
  \begin{center}
    \includegraphics[height=3.9cm]{grave} 
  \end{center} 
  The movement (and the symphony) concludes with an unconventional plagal cadence (subdominant \(\to\) tonic)\footnote{The plagal cadence cowers before the authentic cadence as a resolution; yet its frequent use as an ``Amen'' in many settings have lent it a contrived religious serenity.}. 
\end{frame} 

\begin{frame}
  \begin{center} 
    {\Large\textsc{The End} } 
  \end{center} 
\end{frame} 


\end{document} 
